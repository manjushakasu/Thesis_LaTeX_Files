\chapter{Conclusion and Future Work}
\label{Chapter.seven}

Introducing the concept of intelligent smart agents and agent based modeling in NoC architectures helps in detecting many real time problems. One such problem is excess packet traffic. We created a basic 2D NoC agent based model and a 3D NoC agent based model. A 3D NoC agent based model can monitor the number of packets at each node.

Initially, we have created a 2D NoC agent based model using GeoMason classes. We have created the network and we designed packets which follow XY-routing algorithm to reach the destination location in the network. 

In the 3D NoC agent based model, we designed an environment for the 3D 4x4 mesh network and the basic XYZ routing algorithm. A user can make use of the mesh topology network of the 3D NoC agent based model and develop more complex adaptive routing algorithms. Users should consider the MASON tool for creating agent based models, as MASON can provide duplicable results, that is, simulation results on different machines will remain same when the simulation is carried out using the same parameters. Most of the other simulators are not capable of providing duplicable results. MASON is faster in visualizing models when compared to other simulators. If needed, the models in MASON can be check-pointed and migrated to other machines.

\section{Future Work}
We can implement more complex models using smart agents. Smart agents approach is an evolving concept in the world of artificial intelligence. This could help us to detect, prevent and solve many real time problems in embedded systems. Some problems involved in embedded systems such as excess packet traffic can be prevented by analyzing the systems using smart agents.

In the 3D NoC agent based model we are using scheduling times for the packets to start their scheduling. This can be further modified such that packets start automatically without specifying any scheduling time. In addition, paths are created by reading two nodes each time from an array and drawing a path between them. It is worth designing a set of classes which support the environment creation. Thus far, this model is restricted to basic functionality. Current model can be further extended by developing more complex routing algorithms. 


 
 
  
  
 
 
